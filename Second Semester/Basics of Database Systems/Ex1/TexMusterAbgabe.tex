\documentclass[a4paper]{scrartcl}
\usepackage[a4paper,top=3cm,left=3cm,right=2.5cm]{geometry}
\usepackage[utf8]{inputenc}
\usepackage[naustrian]{babel}
\usepackage[T1]{fontenc}
\usepackage{amsmath,amssymb,enumerate}
\usepackage[donotfixamsmathbugs]{mathtools}
\usepackage{graphicx}
\usepackage{hyperref}
\usepackage{subfig}
\usepackage{float}
\usepackage{mdframed}

\usepackage{scrlayer-scrpage}
\pagestyle{scrheadings}

\newcommand{\join}{\Join}
\newcommand{\lsjoin}{\ltimes}
\newcommand{\rsjoin}{\rtimes}
\newcommand{\ojoin}{{\tiny \textifsym{d|><|d}}}
\newcommand{\lojoin}{{\tiny \textifsym{d|><|}}}
\newcommand{\rojoin}{{\tiny \textifsym{|><|d}}}
% alternativ:
% \newcommand{\ojoin}{=\join=}
% \newcommand{\lojoin}{=\join}
% \newcommand{\rojoin}{\join=}

% Formatierung fuer Primary and Foreign Key
\newcommand{\fk}[1]{\textit{#1}}      % foreign key
\newcommand{\key}[1]{\underline{#1}}  % primary key

% Moegliche Makros fuer Aufgabe 2
\newcommand{\relline}[2]{#1\>(\>#2)\\}
\newcommand{\rellinelo}[1]{ \> \>#1)\\} %left open
\newcommand{\rellinebo}[1]{\> \>#1\\}  %both open
\newcommand{\rellinero}[2]{#1\>(\>#2 \\} %right open


\newcounter{Exernumb}
\newenvironment{exercise}[1]%
   {\stepcounter{Exernumb}\vspace{1em}%
    \setlength{\fboxsep}{3mm}%
    \noindent
    \framebox[\textwidth][l]{%
    \textbf{\textsf{Aufgabe \theExernumb\ (#1)}}%
    \hfill [\hspace{2em} Punkte]%
    }
    \normalfont
   }%
   {%
   }
\newenvironment{subexercises}%
   {\begin{enumerate}[(a)]}%
   {\end{enumerate}}


\newcommand{\ra}{\ensuremath{\rightarrow}}
\newcommand{\la}{\ensuremath{\leftarrow}}

\title{1.~\"Ubungsblatt (WS 2022)}
\subtitle{3.0 VU Datenmodellierung / 6.0 VU Datenbanksysteme}
\author{VNAME, NNAME (Matrikelnr.)}

\automark{section}
\ohead{\pagemark}
\makeatletter
\chead{1.~\"Ubungsblatt DM/DBS (WS 2022) -- \@author}
\makeatother
\cfoot{}

\begin{document}
\maketitle

%%% === WICHTIG!! ANMERKUNG!!! 
%%% Dies ist eine Vorlage(!!) welche helfen soll die Erstellung einer Abgabe zu
%%% erleichtern. Dies ist nicht notwendigerweise ein Ausfuelltext in den einfach
%%% die Antworten eingefuegt werden koennen! Dies bedeutet insbesonders, dass
%%% es auch bei Verwendung dieser Vorlage in Ihrer Verantwortung liegt eine lesbare,
%%% sinnvoll formatierte Abgabe zu erstellen. 

\textbf{ --- HINWEIS!! 
Bitte entfernen Sie diesen Hinweis sowie
s\"amtliche Beispiele welche Sie nicht
gel\"ost haben!! ---}

\section*{EER-Diagramme}


%% AUFGABE 1

\begin{exercise}{EER-Diagramm erstellen}\label{ex:ER}

  Die L\"osung befindet sich in Abbildung~\ref{fig:ER-sol}

  %% kann natuerlich auch direkt eingebunden werden und muss nicht
  %% in einem figure environment stehen.

  \begin{figure}[H]
    % Mit der folgenden Zeile kann eine Grafik eingebunden werden

    % \centering\includegraphics[angle=90,width=\textwidth]{IMAGEHERE}
    \caption{L\"osung zu Aufgabe~\ref{ex:ER}} 
    \label{fig:ER-sol}
  \end{figure}

\end{exercise}




%% AUFGABE 2

\begin{exercise}{Semantik von EER Diagrammen}\label{ex:EERProperties}

  \begin{itemize}
    \item Antwort sollte enthalten: Den Beziehungstyp (Name), welche Notation weggelassen
      wird, eine kurze Erkl\"arung welche Information nicht mehr dargestellt
      werden kann, ein konkretes Beispiel.

  \end{itemize}
    

\end{exercise}




%% AUFGABE 3:
\begin{exercise}{\"Uberf\"uhrung ins Relationenschema}\label{ex:ER-rel}

  %% Natuerlich nur eine Moeglichkeit der Formatierung und
  %% Environments. Kann frei gewaehlt werden.

  \mbox{}
  \begin{quote}\ttfamily
  \begin{tabbing}
      % zur richtigen Einrueckung ("Kreditkarte" funktioniert dieses Smeester gut als Platzhalter fuer den Relationennamen):
      LaengsterRelationenname~\=(\= \kill
      % Eintraege koennen entweder "manuell" angelegt werden
      Relationname \> ( \> \key{SCHL\"USSEL}, \fk{FK}, \key{\fk{KEY \& FK}}, ATTRIBUT) \\

      % oder mittels Makros:
      \relline{Relationenname}{\key{SCHLUESSEL, \fk{KEY \& FK}}} 
      \rellinero{Relationenname}{\key{SCHLUESSEL, \fk{KEY \& FK}},} % "rechts offen": keine ) auf der rechten Seite
      \rellinebo{ATTRIBUT,}                                       % "both offen": kein Tabellenname, kein ( auf der linken Seite,
                                                                    % kein ) auf der rechten Seite
      \rellinelo{ATTRIBUT, \fk{FK}}                               % "links offen": kein Tabellenname, kein ( auf der linken Seite
  \end{tabbing}
  \end{quote}

\end{exercise}






\section*{Aufgaben: Relationale Algebra - Relationenkalk\"ul}

%% AUGABE 4

\begin{exercise}{Auswerten}
  
  %% Wiederum: Nur ein Vorschlag fuer die Formatierung,
  %% darf natuerlich frei gewaehlt werden.


  \begin{subexercises}

    \item % (a) 
      Bitte vergessen Sie nicht das Schema der Ergebnisrelation anzugeben!

    \begin{center}     
      \begin{tabular}{|c|c|c|c|}  %% Hier die richtige Anzahl an Spalten angeben
        \hline
        \multicolumn{4}{|c|}{$q$}\\ %% statt 4 die Anzahl der Spalten (als Zahl) eintragen
        \hline 
        \textbf{AttName1} & \textbf{AttName2} & \textbf{AttName3} & \textbf{AttName4} \\
        \hline
        Wert A1 Tupel 1 & Wert A2 Tupel 1 & Wert A3 Tupel 1 & Wert A4 Tupel 1 \\
        Wert A1 Tupel 2 & Wert A2 Tupel 2 & Wert A3 Tupel 2 & Wert A4 Tupel 2 \\
        %% benoetigte Anzahl an Zeilen eintragen
        \hline  
      \end{tabular}
      \end{center}




    \item % (b)
      Bitte vergessen Sie nicht das Schema der Ergebnisrelation anzugeben!


    \begin{center}     
      \begin{tabular}{|c|c|c|c|}  %% Hier die richtige Anzahl an Spalten angeben
        \hline
        \multicolumn{4}{|c|}{$q$}\\ %% statt 4 die Anzahl der Spalten (als Zahl) eintragen
        \hline 
        \textbf{AttName1} & \textbf{AttName2} & \textbf{AttName3} & \textbf{AttName4} \\
        \hline
        Wert A1 Tupel 1 & Wert A2 Tupel 1 & Wert A3 Tupel 1 & Wert A4 Tupel 1 \\
        Wert A1 Tupel 2 & Wert A2 Tupel 2 & Wert A3 Tupel 2 & Wert A4 Tupel 2 \\
        %% benoetigte Anzahl an Zeilen eintragen
        \hline  
      \end{tabular}
      \end{center}
  \end{subexercises}
  


\end{exercise}




\begin{exercise}{\"Aquivalenzen}

 \noindent \textbf{Aufgabe (a)} \hfill  
  \textbf{Ja/Nein}, $q_1$ und $q_2$ \textbf{sind/sind nicht} \"aquivalent.
 
  Zuerst eine Erkl\"arung warum sie \"aquivalent sind, oder warum nicht.

  Fall sie nicht \"aquivalent sind ein Gegenbeispiel:
 \begin{center}
  \textbf{Gegenbeispiel}
  \vspace{2ex}

  \begin{minipage}{5em}
  \begin{center}
    \begin{tabular}{|c|c|c|}
     \hline
     \multicolumn{3}{|c|}{\textbf{R}} \\
      \hline
      \textbf{A} & \textbf{B} & \textbf{C} \\ \hline
       ? & ? & ? \\ \hline %% so viele Zeilen einsetzen wie ben\"otigt
    \end{tabular}
  \end{center}
  \end{minipage}
  \hspace{1em}
  \begin{minipage}{6em}
  \begin{center}
    \begin{tabular}{|c|c|c|}
     \hline
     \multicolumn{3}{|c|}{\textbf{S}} \\
      \hline
      \textbf{B} & \textbf{D} & \textbf{E} \\ \hline
      ? & ? & ? \\ \hline %% so viele Zeilen einsetzen wie ben\"otigt
    \end{tabular}
  \end{center}
  \end{minipage}
  
  \vspace{1em}

  \begin{minipage}{5em}
  \begin{center}
    \begin{tabular}{|c|c|c|}
     \hline
     \multicolumn{3}{|c|}{\textbf{$q_1$}} \\
      \hline
      \textbf{?} & \textbf{?} & \dots  \\ \hline
       ? & ? & \dots \\ \hline
    \end{tabular}
  \end{center}
  \end{minipage}
  \hspace{1em}
  \begin{minipage}{6em}
  \begin{center}
    \begin{tabular}{|c|c|c|}
     \hline
     \multicolumn{3}{|c|}{\textbf{$q_2$}} \\
      \hline
      \textbf{?} & \textbf{?} & \dots  \\ \hline
       ? & ? & \dots \\ \hline
    \end{tabular}
  \end{center}
  \end{minipage}
 \end{center}
\end{exercise}



\begin{exercise}{Gr\"o{\ss}enabsch\"atzung}

  \noindent \textbf{Aufgabe (a)} \hfill  [Minimum: M | Maximum: M]

  Begr\"undung (sowohl Minimum als auch Maximum)

 \begin{center}
  \textbf{Minimum: ??}
  \vspace{2ex}

  \begin{minipage}{8em}
  \begin{center}
    \begin{tabular}{|c|c|c|c|}
     \hline
     \multicolumn{4}{|c|}{\textbf{S}} \\
      \hline
      \textbf{A} & \textbf{B} & \textbf{\underline{C}} & \textbf{\underline{D}} \\ \hline
      ? & ? & ? & ? \\ \hline
      ? & ? & ? & ? \\ \hline
      ? & ? & ? & ? \\ \hline
      ? & ? & ? & ? \\ \hline
      ? & ? & ? & ? \\ \hline
      ? & ? & ? & ? \\ \hline
      ? & ? & ? & ? \\ \hline
    \end{tabular}
  \end{center}
  \end{minipage}
  \hspace{1em}
  \begin{minipage}{5em}
  \begin{center}
    \begin{tabular}{|c|c|c|}
     \hline
     \multicolumn{3}{|c|}{\textbf{T}} \\
      \hline
      \textbf{A} & \textbf{C} & \textbf{\underline{E}} \\ \hline
      ? & ? & ? \\ \hline
      ? & ? & ? \\ \hline
      ? & ? & ? \\ \hline
      ? & ? & ? \\ \hline
    \end{tabular}
  \end{center}
  \end{minipage}
  \hspace{5em}
  \begin{minipage}{3em}
  \begin{center}
    \begin{tabular}{|c|c|}
     \hline
     \multicolumn{2}{|c|}{\textbf{Ergebnis}} \\
      \hline
      \textbf{E} & \textbf{X} \\ \hline
      ? & ? \\ \hline
    \end{tabular}
  \end{center}
  \end{minipage}
 \end{center}
 
 \bigskip

 \begin{center}
  \textbf{Maximum: ??}
  \vspace{2ex}

  \begin{minipage}{8em}
  \begin{center}
    \begin{tabular}{|c|c|c|c|}
     \hline
     \multicolumn{4}{|c|}{\textbf{S}} \\
      \hline
      \textbf{A} & \textbf{B} & \textbf{\underline{C}} & \textbf{\underline{D}} \\ \hline
      ? & ? & ? & ? \\ \hline
      ? & ? & ? & ? \\ \hline
      ? & ? & ? & ? \\ \hline
      ? & ? & ? & ? \\ \hline
      ? & ? & ? & ? \\ \hline
      ? & ? & ? & ? \\ \hline
      ? & ? & ? & ? \\ \hline
    \end{tabular}
  \end{center}
  \end{minipage}
  \hspace{1em}
  \begin{minipage}{5em}
  \begin{center}
    \begin{tabular}{|c|c|c|}
     \hline
     \multicolumn{3}{|c|}{\textbf{T}} \\
      \hline
      \textbf{A} & \textbf{C} & \textbf{\underline{E}} \\ \hline
      ? & ? & ? \\ \hline
      ? & ? & ? \\ \hline
      ? & ? & ? \\ \hline
      ? & ? & ? \\ \hline
    \end{tabular}
  \end{center}
  \end{minipage}
  \hspace{5em}
  \begin{minipage}{3em}
  \begin{center}
    \begin{tabular}{|c|c|}
     \hline
     \multicolumn{2}{|c|}{\textbf{Ergebnis}} \\
      \hline
      \textbf{E} & \textbf{X} \\ \hline
      ? & ? \\ \hline
    \end{tabular}
  \end{center}
  \end{minipage}
 \end{center}


\end{exercise}

\vspace{1em}


\begin{exercise}{Abfragesprachen}

  \begin{subexercises}
    \item Für Abfrage gegben in Relationaler Algebra:

      übersetzt in \textbf{Tupelkalk\"ul:}

     übersetzt in \textbf{Dom\"anenkalk\"ul:}

    \item Für Abfrage gegeben in Domänenkalkül

      übersetzt in \textbf{Tupelkalk\"ul:}
      
      übersetzt in  \textbf{Relationale Algebra:}

    \item Für Abfrage gegeben in Domänenkalkül

      übersetzt in \textbf{Tupelkalk\"ul:}

      übersetzt in  \textbf{Relationale Algebra:}

  \end{subexercises}  

\end{exercise}



\begin{exercise}{Formalisieren von Anfragen}\label{ex:ra-formalisieren}

  \begin{subexercises}
    \item 

    \centerline{\textbf{Relationale Algebra:}}

    \[
      Relationaler Algebra Ausdruck 
    \]
   
    \medskip
    \centerline{\textbf{Tupelkalk\"ul:}}
    \vspace{-.8cm}
    \begin{multline*}
      \{ Ausgabe  \mid Definition \}
  \end{multline*}
  
  \medskip
    \centerline{\textbf{Dom\"anenkalk\"ul:}}
    \vspace{-.8cm}
    \begin{multline*}
      \{ Ausgabe  \mid Definition \}
    \end{multline*}


    \item 

    \centerline{\textbf{Relationale Algebra:}}
    \[
      Relationaler Algebra Ausdruck 
    \]
   
    \medskip
    \centerline{\textbf{Tupelkalk\"ul:}}
    \vspace{-.8cm}
    \begin{multline*}
      \{ Ausgabe  \mid Definition \}
    \end{multline*}
  
  \medskip
    \centerline{\textbf{Dom\"anenkalk\"ul:}}
    \vspace{-.8cm}
    \begin{multline*}
      \{ Ausgabe  \mid Definition \}
    \end{multline*}


    \item 

    \centerline{\textbf{Relationale Algebra:}}
    \[
      Relationaler Algebra Ausdruck 
    \]
   
    \medskip
    \centerline{\textbf{Tupelkalk\"ul:}}
    \vspace{-.8cm}
    \begin{multline*}
      \{ Ausgabe  \mid Definition \}
  \end{multline*}
  
  \medskip
    \centerline{\textbf{Dom\"anenkalk\"ul:}}
    \vspace{-.8cm}
    \begin{multline*}
      \{ Ausgabe  \mid Definition \}
    \end{multline*}


      \item 

    \centerline{\textbf{Relationale Algebra:}}
    \[
      Relationaler Algebra Ausdruck 
    \]
   
    \medskip
    \centerline{\textbf{Tupelkalk\"ul:}}
    \vspace{-.8cm}
    \begin{multline*}
      \{ Ausgabe  \mid Definition \}
  \end{multline*}
  
  \medskip
    \centerline{\textbf{Dom\"anenkalk\"ul:}}
    \vspace{-.8cm}
    \begin{multline*}
      \{ Ausgabe  \mid Definition \}
    \end{multline*}

  \end{subexercises}
  
\end{exercise}


\end{document}

